\documentclass[10pt,]{article}
\usepackage[left=1in,top=1in,right=1in,bottom=1in]{geometry}
\newcommand*{\authorfont}{\fontfamily{phv}\selectfont}
\usepackage[]{mathpazo}


  \usepackage[T1]{fontenc}
  \usepackage[utf8]{inputenc}



\usepackage{abstract}
\renewcommand{\abstractname}{}    % clear the title
\renewcommand{\absnamepos}{empty} % originally center

\renewenvironment{abstract}
 {{%
    \setlength{\leftmargin}{0mm}
    \setlength{\rightmargin}{\leftmargin}%
  }%
  \relax}
 {\endlist}

\makeatletter
\def\@maketitle{%
  \newpage
%  \null
%  \vskip 2em%
%  \begin{center}%
  \let \footnote \thanks
    {\fontsize{18}{20}\selectfont\raggedright  \setlength{\parindent}{0pt} \@title \par}%
}
%\fi
\makeatother




\setcounter{secnumdepth}{0}



\title{Introduction  }



\author{\Large Matt Barreto\vspace{0.05in} \newline\normalsize\emph{University of California-Los Angeles}   \and \Large Marcel Roman\vspace{0.05in} \newline\normalsize\emph{University of California-Los Angeles}   \and \Large Hannah Walker\vspace{0.05in} \newline\normalsize\emph{Rutgers University}  }


\date{}

\usepackage{titlesec}

\titleformat*{\section}{\normalsize\bfseries}
\titleformat*{\subsection}{\normalsize\itshape}
\titleformat*{\subsubsection}{\normalsize\itshape}
\titleformat*{\paragraph}{\normalsize\itshape}
\titleformat*{\subparagraph}{\normalsize\itshape}


\usepackage{natbib}
\bibliographystyle{apsr}
\usepackage[strings]{underscore} % protect underscores in most circumstances



\newtheorem{hypothesis}{Hypothesis}
\usepackage{setspace}

\makeatletter
\@ifpackageloaded{hyperref}{}{%
\ifxetex
  \PassOptionsToPackage{hyphens}{url}\usepackage[setpagesize=false, % page size defined by xetex
              unicode=false, % unicode breaks when used with xetex
              xetex]{hyperref}
\else
  \PassOptionsToPackage{hyphens}{url}\usepackage[unicode=true]{hyperref}
\fi
}

\@ifpackageloaded{color}{
    \PassOptionsToPackage{usenames,dvipsnames}{color}
}{%
    \usepackage[usenames,dvipsnames]{color}
}
\makeatother
\hypersetup{breaklinks=true,
            bookmarks=true,
            pdfauthor={Matt Barreto (University of California-Los Angeles) and Marcel Roman (University of California-Los Angeles) and Hannah Walker (Rutgers University)},
             pdfkeywords = {Immigration, Law Enforcement, Political Participation, Policy Feedback},  
            pdftitle={Introduction},
            colorlinks=true,
            citecolor=blue,
            urlcolor=blue,
            linkcolor=magenta,
            pdfborder={0 0 0}}
\urlstyle{same}  % don't use monospace font for urls

% set default figure placement to htbp
\makeatletter
\def\fps@figure{htbp}
\makeatother



% add tightlist ----------
\providecommand{\tightlist}{%
\setlength{\itemsep}{0pt}\setlength{\parskip}{0pt}}

\begin{document}
	
% \pagenumbering{arabic}% resets `page` counter to 1 
%
% \maketitle

{% \usefont{T1}{pnc}{m}{n}
\setlength{\parindent}{0pt}
\thispagestyle{plain}
{\fontsize{18}{20}\selectfont\raggedright 
\maketitle  % title \par  

}

{
   \vskip 13.5pt\relax \normalsize\fontsize{11}{12} 
\textbf{\authorfont Matt Barreto} \hskip 15pt \emph{\small University of California-Los Angeles}   \par \textbf{\authorfont Marcel Roman} \hskip 15pt \emph{\small University of California-Los Angeles}   \par \textbf{\authorfont Hannah Walker} \hskip 15pt \emph{\small Rutgers University}   

}

}








\begin{abstract}

    \hbox{\vrule height .2pt width 39.14pc}

    \vskip 8.5pt % \small 

\noindent This is an introduction to the WPSA/MPSA paper we are presenting on the
effects of contact with imimgration enforcement on political
participation and policy attitudes.


\vskip 8.5pt \noindent \emph{Keywords}: Immigration, Law Enforcement, Political Participation, Policy Feedback \par

    \hbox{\vrule height .2pt width 39.14pc}



\end{abstract}


\vskip 6.5pt


\noindent \singlespacing \pagebreak

\doublespacing

\pagebreak

\section{Introduction}\label{introduction}

What are the effects of immigration enforcement on political
participation and evaluations of immigration policy prescriptions? The
extant literature is mixed. Some research suggests that immigration
enforcement can have a mobilizing effect on various forms of electoral
and non-electoral political behavior
\citep{corneliusdeath2001, pantojacitizens2001, whitewhen2016, zepeda-millanlatino2017}.
Other research indicates that immigration enforcement demobilizes
political participation and depresses engagement with the state,
particularly amongst non-citizens and Latinx
\citep{watsoninside2014, pedrazaimmigration2015, vargasimmigration2015, vargasmixed-status2016, potochnicklocal-level2017, amuedo-dorantesinterior2017}.
However, despite work on the link between proximate exposure to
non-immigration related law enforcement and political participation
\citep{leeconsequences2014, walkerextending2014, walkerfor2017, walkertargeted:nodate},\footnote{Proximate
  contact or exposure meaning that an individual knows family members
  and/or friends that are or have been exposed to law enforcement. There
  is some research on the effects of knowing someone undocumented on
  political participation \citep{streetpolitical2017}, but this paper is
  distinct in that a) it also evaluates the effects of proximate contact
  with immigration authorities, not just knowing someone undocumented
  and b) it analyzes not only Latinx populations, but the effects of
  proximate contact on other racial/ethnic groups.} left unanswered in
the literature is an investigation of the effects of proximate contact
with immmigration enforcement on both various types of political
participation \emph{and} attitudes regarding immigration policy
prescriptions.\footnote{Although there is a rich literature on the
  effects of contact with the criminal justice system on public trust
  and perceptions of the criminal justice system writ large
  \citep{tylertrust2002, weitzerincidents2002, engelcitizens2005, peffleyjustice2010, lermanarresting2014}.}
Moreover, there is little attention paid to the mechanisms that motivate
individual behavior in response to contact with immigration
enforcement.\footnote{For an exception, see \citet{whitewhen2016}, who
  leverages CCES data and finds that the reason Latinx are mobilized
  after the implementation of anti-immigrant policy has more to do with
  mobilization by parties instead of direct responses to the policy
  threat.}

This paper fills in the gap by leveraging a survey deployed in 2018 with
representative samples of several different racial and ethnic groups to
investigate whether or not proximate exposure to immigration enforcement
depresses or increases various forms of political participation and if
these relationships vary by race and ethnicity. Since the survey
instrument included several questions regarding evaluations of
immigration policy prescriptions, this paper also provides an assessment
of potential policy feedback effects from immigration enforcement on
public opinion over immigration policy. In addition, in order to reveal
potential individual-level mechanisms that may be motivating political
behavior in response to proximate contact, this paper investigates both
whether or not proximate exposure influences a sense of injustice and
how a sense of injustice affects both political participation and
immigration policy preferences. Analyzing the link between perceptions
of injustice and proximate contact is important since previous research
finds that perceptions of injustice and discrimination with respect to
the criminal justice system influence how family members and friends
respond to proximate exposure \citep{walkertargeted:nodate}.

The results indicate that, across all racial categories, proximate
contact with immigration enforcement mobilizes non-traditional forms of
political participation such as protesting or attending demonstrations,
but does not have an effect on traditional forms of participation such
as voting. This paper identifies a potential mechanism in showing that
non-traditional mobilization may be driven by a perception of an unjust
legal and immigration system. Moreover, this paper finds that proximate
contact with immigration enforcement may generate changes in how the
public perceives immigration policy prescriptions. Proximate contact is
associated with a progressive shift in evaluating immigration policy
proposals such as comprehensive immigration reform and even abolishing
ICE.

Additionally, this paper finds some racial heterogeneity in the effects
of proximate contact with immigration enforcement. Interestingly, there
are no heterogenous effects by racial or ethnic group category with
respect to proximate contact with immigration authorities and political
protest, which contradicts previous work on the racially heterogenous
effects of exposure to the criminal justice system
\citep{walkerextending2014, walkertargeted:nodate}. All groups are
equally mobilized by proximate exposure to immigration
enforcement.\footnote{Albeit the point estimates are larger for
  non-white groups, with the upper bounds for non-white groups exceeding
  the upper bounds for whites (with the exception of AAPI).} However,
consistent with theories in the extant literature that suggest that
targeted populations are more likely to respond politically in the face
of policy threat than non-targeted populations
\citep{pantojafear2003, kangelectoral2017, laniyonupolitical2018}, this
paper finds that Latinx are more likely to shift in a progressive
direction with respect to evaluations of immigration policy
prescriptions than other racial group categories.

This paper makes several contributions. First, given the nature of the
sample, there is sufficient statistical power for a cross-racial/ethnic
group analysis.\footnote{With 400 respondents for the 5 relevant
  racial/ethnic group categories: white, black, Latinx, Native American,
  and AAPI.} Therefore, the sample provides a unique opportunity to
study the effects of immigration enforcement on groups that are not only
politically underrepresented, but underrepresented when it comes to
research on the effects of enforcement, such as Native Americans and
Asians. Second, the research question moves beyond merely
conceptualizing criminal justice through the policing of black American
communities and individuals (e.g.~what some scholars would call the
``black/white'' binary, although this understanding is somewhat
problematic in the context of immigration policy given the existence of
foreign-born black people and black immigrants). Understanding law
enforcement through the lens of immigration policy is particularly
important given that there are a disproportionate amount of resources
devoted to immigration enforcement at the Federal level,\footnote{In
  2012,
  \href{https://www.theatlantic.com/magazine/archive/2018/09/trump-ice/565772/}{Congress
  appropriated \$18 billion for immigration enforcement while it
  appropriated \$14 billion for all the other major criminal law
  enforcement agencies combined.}} it is an increasingly salient issue
given the election of Donald Trump, and the U.S. is undergoing a
demographic transformation in that the percentage of the foreign-born
population is at its highest since 1910.\footnote{According to the
  Migration Policy Institute,
  \href{https://www.migrationpolicy.org/article/immigrants-us-states-fastest-growing-foreign-born-populations}{``The
  foreign-born share of the U.S. population is at its highest level
  since 1910, with the approximately 44 million immigrants living in the
  United States representing 13.5 percent of the overall
  population\ldots{}.With U.S. fertility rates at a historic low, the
  Census Bureau projects that net international migration will be the
  main driver behind U.S. population growth between 2027 and 2038.''}}
Third, while there is some research to suggest that criminal justice
policies affect how individuals judge the state, legal system, and
criminal justice policy writ large, there is limited research on the
effects of immigration enforcement on the evaluation of immigration
policy prescriptions. This paper fills in the gap by employing a survey
asking respondents to evaluate various proposals that may offer a
solution to the immigration status of millions of people residing in the
U.S. such as conditional residency for alien minors (the DREAM act),
comprehensive immigration reform (CIR), and even abolishing ICE. Fourth,
while there is some work on the de(mobilizing) effects of immigration
enforcement, much of it is limited to analyzing electoral outcomes such
as voting and registration
\citep{whitewhen2016, amuedo-dorantesinterior2017}. Moreover, much of
the work analyzing the determinants of protest activity related to
anti-immigrant policies does not take into account prior contact with
immigration enforcement officers \citep{zepeda-millanlatino2017}. This
paper fills in this gap, both by analyzing non-electoral political
activity as an outcome and accounting for proximate exposure to contact
with immigration enforcement officials.
\newpage
\singlespacing 
\renewcommand\refname{References}
\bibliography{references.bib}
\end{document}
